\section{Approach}
To schedule processing tasks on software defined networks, we identified two approaches, named
Top-Down and Bottom-Up. A sketch of these concepts and their relation to the setup is depicted below
and described in the following section.\\

We make the following assumptions:
\begin{itemize}
\item Jobs are not scheduled in parallel but one after another. Otherwise, the scheduler may interfere
with scheduling multiple jobs concurrently.

\item Jobs always communicate All-To-All. This holds true for the experiments conducted within this paper
and eliminates the need to traverse the execution graph. Also, this benefits the generic approach
since no deep Flink requirement is built up.

\item Due to the fact that network behaviour is to be studied, the focus lies on network communication. If
a single task manager offers multiple processing slots, network effects are weakened. Therefore, a
task manager consists of a single processing slot in our setup.
\end{itemize}

These constraints limit real-world applications but can easily hold true for our test setup to
determine performance gains with the concepts introduced.

\subsection{Top-Down/Flow Switching}
The Top-Down approach stipulates that the data processing engine schedules tasks as usual and does
not take network properties into account. The middleware is aware of these scheduling decisions and
influences the network properly. This influence could be either providing additional bandwidth
between communicating hosts through flow switching or limiting bandwidth on the specific
communication links for background activities. Advantages of this approach are that there is no
modification on the scheduler required and it is more portable since only a data processing
engine-middleware interface must be implemented. A major disadvantage is the possible worst case:
Parts of a job could be scheduled in a data center and another part far away like in another region
or colocation.

\subsection{Bottom-Up/Scheduling}
The Bottom-Up approach envisages that the data processing engines’ scheduler actively asks the
middleware for a node to place the task on. The middleware is aware of the network topology and can
determine optimal task placement through proper algorithms. The disadvantage of this approach is
that the data processing engines’ scheduler needs to be modified to interact with the middleware.
Advantages would be that the middleware is in full control of processing job placement what not only
guarantees optimal placement but also can add optional constraints like only use certain network
areas for processing during maintenance etc. 

A detailed description of the actual implemented algorithms used is given in section TODO
